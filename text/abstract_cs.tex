\begin{mtabstract}[Abstrakt]

\selectlanguage{czech}
Cílem této diplomové práce je implementace nástrojů pro vizualizaci
časo\-pros\-toro\-vých dat v geografickém informačním systému GRASS.
Tyto nástroje využívají nedávno vyvinutého rozšíření GRASSu,
GRASS GIS Temporal Framework, které je určeno pro správu, zpracování a analýzu
časoprostorových dat. Tři nové aplikace byly
vytvořeny pomocí grafické knihovny wxPython a byly začleněny do systému GRASS.
Mezi tyto aplikace patří nástroje pro animaci map, interaktivní porovnávání map
a vizualizaci metadat časoprostorových datasetů.
Teoretická část diplomové práce se zabývá časoprostorovými daty obecně,
a dále pak způsoby jejich vizualizace. Obsahuje také seznam současných softwarových
projektů, které jsou schopné zpracovávat časoprostorová data.


\bigskip
\bigskip
\bigskip
\bigskip


\keyvalue{Klíčová slova}{GRASS, GIS, časoprostorová data, vizualizace}
\selectlanguage{english}
\end{mtabstract}
