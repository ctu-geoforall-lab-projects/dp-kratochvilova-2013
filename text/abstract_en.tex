\begin{mtabstract}
The aim of this master thesis was to implement software tools for visualization
of spatio-temporal data in GRASS GIS. These tools make use of the recent addition
to GRASS 7, the GRASS GIS \tf which has been developed to manage, process and analyze
large scale, spatio-temporal environmental data.
Three new tools have been implemented, using the GUI toolkit wxPython, and incorporated into GRASS 7.
These applications include map animation, interactive comparison of two maps by ``swiping''
and visualization of temporal datasets' metadata.
The theoretical part of the thesis deals with spatio-temporal data in general, with special
focus on visualization approaches. List of currently available software
capable of handling spatio-temporal data is included.





\bigskip
\bigskip
\bigskip
\bigskip

\keyvalue{Key words}{GRASS, GIS, spatio-temporal data, visualization}

\end{mtabstract}
