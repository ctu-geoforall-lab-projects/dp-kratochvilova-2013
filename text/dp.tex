\documentclass[a4paper,12pt]{article}
\usepackage[utf8x]{inputenc}
\usepackage[english]{babel}
\usepackage{a4wide}
\usepackage{url}
\usepackage{multirow}
\usepackage{array}
\usepackage{booktabs}
\usepackage{subfigure}


%opening
\title{}
\author{Anna Kratochvílová}

\newcommand{\intervals}[4]{%
\begin{minipage}[c]{6cm}
X \hspace{#1} \rule[3pt]{#2}{1mm} \\
Y \hspace{#3} \rule[3pt]{#4}{1mm}
 \end{minipage}
 }

 
\begin{document}


\section{Temporal GRASS GIS framework}
Temporal GRASS GIS Framework is a new extension available in GRASS 7 for manipulating spatio-temporal data.
It enables to manage, analyze and process large amount of spatio-temporal data.
Following the GRASS GIS modular design, the framework introduces over 30 new modules.
Temporal modules' names starting with t./t.rast/t.vect/t.rast3d names
comply with the GRASS modules' naming conventions.

\subsection{Implementation}
Temporal framework uses a snapshot approach (described above).
This approach can be easily understood and is simple enough to be integrated into the layer-based GIS.
The integration consists of two levels.
In the first level, timestamps are assigned to the existing spatial data types -- raster, vector, 3D raster maps.
The second level introduces new datatypes -- space time raster,
vector and 3D raster datasets (referred as stdrs, stvds and str3ds).

The implementation of the temporal framework is based on a temporal library and a SQL database scheme.
The library is written in Python programming language and provides an application programming interface (API) which is used by the temporal modules,
however the library is meant to be used not only by other modules but also within user scripts or the GRASS GUI if needed.
Beside valid time the temporal database stores also transaction time%
\footnote{Valid time is the time period during which a database fact is valid in the modeled reality,
transaction time  is the time period during which a database fact is stored in the database \cite{temporalGlossary}}
and other map and dataset metadata.
The temporal framework supports two different database back-ends -- SQLite%
\footnote{\url{http://www.sqlite.org/}} (more lightweight) and
PostgreSQL\footnote{\url{http://www.postgresql.org/}}.
However, GRASS GIS users usually do not have to come in contact with the underlying database backend
as the default SQLite driver is often what they need.

\subsection{Basic concepts}
The temporal framework follows the concept of linear and discrete time.
To understand the basic concept of temporal framework it is necessary to explain several terms.
The terms are explained in the context of the temporal framework, more theoretical and accurate definitions can be found in \cite{temporalGlossary}.

\paragraph{Interval vs. point time}
\label{sec:intervalVsPoint}
One can decide which time model to use -- interval time or time instances (also called point time).
Point time is a single moment in the time dimension
while interval is a period of time consisting of two instances -- start time instance and end time instance.
The time interval contains the start time but not the end time:
$$[start, end)$$

Each type is suitable for different types of data.
Consider temperature and precipitation measuring.
While temperature is measured in a given time instance and the measured value describes the state,
precipitation is measured over a given time period. The decision which model to choose is not always
straightforward, however it appears that for many applications,
interval time is a better choice \cite{pointVsInterval}.

\paragraph{Absolute vs. relative time}
\label{sec:absoluteVsRelative}
Absolute time stamp is related to a fact and it does not depend on any other facts.
On the contrary, relative time is related to another time and
it can be represented even by a negative number which stands for ``before".
The temporal framework recognizes several date time formats for absolute time -- TODO.
Relative time format consists of a number and time unit, which can be one the following: \emph{years},
\emph{months}, \emph{days}, \emph{hours}, \emph{minutes} and \emph{seconds} (see table \ref{tab:timeFormat}).

\begin{table}[h!]
  \centering
\setlength{\extrarowheight}{3pt}
\begin{tabular}{|l|l|}
\hline
\multirow{3}{*}{Absolute time}
 & 2001-01-05  \\ \cline{2-2}
 & 1996-07-06 23:01:59  \\\cline{2-2}
 & TODO  \\
 \hline
\multirow{2}{*}{Relative time}
 & 1 months  \\\cline{2-2}
 & -5 years  \\
 \hline
\end{tabular}

  \caption{Examples of valid time formats}
  \label{tab:timeFormat}
\end{table}

\paragraph{Temporal granularity}
\label{sec:temporalGranularity}
Temporal granularity is an important characteristics of every temporal dataset.
In the context of the temporal framework, it represents the greatest common divisor
of the temporal extents (and possible gaps) of all maps of the dataset.
The temporal granularity can change every time there is a change in dataset maps
(added, removed or changed time stamp).
To understand better, see table \ref{tab:granularity} content of which is based on the output
of one of the temporal modules (namely \emph{t.rast.list}).

Example interval time dataset consists of 6 maps and
there are also two gaps which means that at this time period no data are available.
The column \emph{Duration} contains the number of days for each period and it can be inferred
that the greatest common divisor is 2 months.
\begin{table}[h!]
  \centering
\setlength{\extrarowheight}{3pt}
\begin{tabular}{cccr}
\toprule
 Map name & Start time & End time & Duration [days]\\\midrule
avg\_temp.01@timeseries &    2001-01-01 00:00:00  &   2001-03-01 00:00:00  &   59.0\\
avg\_temp.02@timeseries  &  2001-03-01 00:00:00  &   2001-05-01 00:00:00   &  61.0\\
no map (gap)    &  2001-05-01 00:00:00    & 2001-09-01 00:00:00   &  123.0\\
avg\_temp.03@timeseries     & 2001-09-01 00:00:00   &  2001-11-01 00:00:00  &   61.0\\
no map (gap) &   2001-11-01 00:00:00   &  2002-01-01 00:00:00  &   61.0 \\
avg\_temp.04@timeseries  &    2002-01-01 00:00:00   &  2002-05-01 00:00:00     & 120.0 \\
avg\_temp.05@timeseries  &      2002-05-01 00:00:00 &    2002-07-01 00:00:00 &    61.0 \\
avg\_temp.06@timeseries &     2002-07-01 00:00:00  &   2002-09-01 00:00:00   &  62.0\\
\bottomrule
\end{tabular}

  \caption{Example dataset}
  \label{tab:granularity}
\end{table}


\paragraph{Temporal topology}
\label{sec:temporalTopology}
Although topology is a term widely used in the area of mathematics studying properties of space,
this term can be used analogically for the time dimension.
Temporal topology analyzes temporal relations between time stamps (for both interval and point time).
There are 13 base relations between two intervals according to \cite{relationships}.
Table \ref{tab:relationships} shows these relations (without the inverse relations).


\begin{table}[h]
\begin{center}
\setlength{\extrarowheight}{10pt}

% \setlength{\unitlength}{5cm}
% \linethickness{5mm}
\begin{tabular}{|p{6.5cm}|l|}

\hline
\intervals{0cm}{2cm}{3cm}{2cm} \vspace{5pt} &  X before Y   \\\hline
\intervals{0cm}{2cm}{2cm}{2cm} \vspace{5pt} &  X meets Y \\\hline
\intervals{0cm}{3cm}{2cm}{3cm} \vspace{5pt} &  X overlaps with Y  \\\hline
\intervals{0cm}{3cm}{0cm}{5cm} \vspace{5pt} &  X starts Y  \\\hline
\intervals{1cm}{3cm}{0cm}{5cm} \vspace{5pt} &  X during Y  \\\hline
\intervals{2cm}{3cm}{0cm}{5cm} \vspace{5pt} &  X ends Y  \\\hline
\intervals{0cm}{5cm}{0cm}{5cm} \vspace{5pt} &  X equal Y   \\\hline

\end{tabular}
\end{center}
\caption{Temporal relationships according to \cite{relationships}}
\label{tab:relationships}
\end{table}

The topology of a temporal dataset can be valid or invalid which depends on the relationship between dataset maps.
If certain maps overlap or one is contained by the other then the dataset has invalid topology.
As a consequence, such dataset is not accepted by certain temporal modules.

\paragraph{Temporal sampling}
\label{sec:temporalSampling}
Temporal sampling is used to determine the state of one process during a second process.
Temporal framework enables to sample a dataset (can have both point and interval time) by another dataset having interval time.
There is an example of temporal sampling in figure \ref{fig:samplingExample}.
The figure indicates that for different sampling methods (temporal relationships)
which are provided by module \emph{t.sample}, different results are expected.

In figure \ref{fig:samplingExampleReverse}, the input (sampled) dataset and the sample dataset are interchanged
to demonstrate the reciprocity of the temporal relations. For example, the table \ref{fig:samplingTable} shows that
there is a relation \emph{during} between intervals $Y_2$ and $X_2$, $X_3$.
When the same datasets are interchanged the relation must remain, however it is transformed into \emph{contain}.
The same applies for relations \emph{precede} and \emph{follow}.
Relation \emph{equal} does not have any opposite relation and \emph{overlap} involves both cases (\emph{overlap, overlapped}).
The opposite of the relation \emph{start} is \emph{end} which is not included in t.sample options.



\begin{figure}[ht]
\centering
\subfigure[Space time datasets with interval time]{
\setlength{\unitlength}{1cm}
  \begin{tabular}{ll}
    Sampled dataset & \framebox[3cm][c]{$X_1$}\framebox[1cm][c]{$X_2$}\framebox[2cm][c]{$X_3$}\framebox[3cm][c]{$X_4$} \\
    & \\
    Sample dataset & \rule{1cm}{0cm}\framebox[1cm][c]{$Y_1$}\framebox[4cm][c]{$Y_2$}\framebox[3cm][c]{$Y_3$} \\
    & \hspace{1cm}\raisebox{3pt}{\thicklines \vector(1, 0){5}} time \\
    & \\
  \end{tabular}
  \label{fig:samplingDatasets}
}
\subfigure[Sampling result for each temporal relationship]{
\setlength{\extrarowheight}{3pt}
    \begin{tabular}{c|c|c|c|c|c|c|c|}
    & start & during & contain & overlap & equal &follow &precede\\\hline
    $Y_1$ & --- & --- & $X_1$ & --- & ---&--- &---\\
    $Y_2$ & $X_2$, $X_3$ & $X_2$, $X_3$ & --- & $X_1$ & ---& $X_4$&--- \\
    $Y_3$ & $X_4$ &---  & --- & --- & $X_4$& ---&$X_3$
    \end{tabular}
  \label{fig:samplingTable}
}
\caption{Example of space time dataset sampling }
\label{fig:samplingExample}
\end{figure}


\begin{figure}[ht]
\centering
\subfigure[Space time datasets with interval time (switched datasets from figure \ref{fig:samplingExample})]{
\setlength{\unitlength}{1cm}
  \begin{tabular}{ll}
    Sampled dataset & \rule{1cm}{0cm}\framebox[1cm][c]{$Y_1$}\framebox[4cm][c]{$Y_2$}\framebox[3cm][c]{$Y_3$} \\
    & \\
    Sample dataset & \framebox[3cm][c]{$X_1$}\framebox[1cm][c]{$X_2$}\framebox[2cm][c]{$X_3$}\framebox[3cm][c]{$X_4$}\\
    & \hspace{1cm}\raisebox{3pt}{\thicklines \vector(1, 0){5}} time \\
    & \\
  \end{tabular}
  \label{fig:samplingDatasetsReverse}
}
\subfigure[Sampling result for each temporal relationship]{
\setlength{\extrarowheight}{3pt}
    \begin{tabular}{c|c|c|c|c|c|c|c|}
          & start        & during & contain & overlap & equal &follow &precede\\\hline
    $X_1$ & $Y_1$, $Y_2$ & $Y_1$  & ---     & $Y_2$   & ---   &---    &---\\
    $X_2$ & ---          & ---    & $Y_2$   & ---     & ---   & ---   &--- \\
    $X_3$ & ---          & ---    & $Y_2$   & ---     & ---   & $Y_3$ &--- \\
    $X_4$ & $Y_3$        &---     & ---     & ---     & $Y_3$ & ---   &$Y_2$
    \end{tabular}
  \label{fig:samplingTableReverse}
}
\caption{Example of space time dataset sampling (switched sampled and sample dataset)}
\label{fig:samplingExampleReverse}
\end{figure}



\subsection{Functionality}
The temporal library provides wide functionality which is accessible either via Python API or GRASS temporal modules.
The main capabilities of the library include:

\begin{itemize}
    \item Registration of raster, vector a 3D raster maps in datasets (called \emph{strds}, \emph{stvds}, \emph{str3ds}, respectively).
    \item Support of absolute and relative time (only maps of one type can be in one dataset).
    \item Support of interval and point time (can be used together in one dataset).
    \item Metadata (number of maps, spatial, temporal extent, creation time, \ldots) of a dataset
    are recomputed automatically after (un)registering of maps.
    \item Computation of time granularity.
    \item Computation of temporal topology and reporting its validity.
    \item Temporal sampling of one dataset by another dataset or by its time granularity.
\end{itemize}

Other functionality needed for handling spatio-temporal data is provided by temporal modules.
The integration of the framework into GRASS GIS enables the temporal framework to reuse existing
functionality to process spatial data.

Temporal modules share certain common options:
\begin{itemize}
  \item They allow to specify the temporal range of datasets so that only a part of a dataset can be processed.
  \item Several modules support two possibilities of specifying the input data\,---\,directly or by providing a file.
  \item When modules output a report or a table it is possible to change the format suitably.
\end{itemize}

The important ability of temporal framework is the interoperability
with several powerful open-source applications for statistical computations and visualization.
Textual ouputs of many modules are suitable for statistical environment \emph{R}\footnote{\url{www.r-project.org/}}.
Module \emph{r3.out.netcdf} exports 3D raster to NetCDF
format\footnote{\url{http://www.opengeospatial.org/standards/netcdf}}
which can be then processed e.g. in \emph{Climate Data Operator (CDO)}\footnote{\url{https://code.zmaw.de/projects/cdo}}.
The support of visualization software \emph{ParaView}\footnote{\url{http://www.paraview.org/}} is available
through VTK format which can be exported by temporal module \emph{t.rast.out.vtk}.




% TODO: space time voxel cubes

















\newpage
\clearpage
\bibliographystyle{ieeetr}
\bibliography{./references.bib}

\end{document}
