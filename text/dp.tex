\documentclass[a4paper,12pt]{article}
\usepackage[utf8x]{inputenc}
\usepackage{a4wide}
\usepackage{url}

%opening
\title{}
\author{Anna Kratochvílová}

\begin{document}


\section{Temporal GRASS GIS framework}
Temporal GRASS GIS Framework is a new extension available in GRASS 7 for manipulating spatio-temporal data.
It enables to manage, analyze and process large amount of spatio-temporal data.
Following the GRASS GIS modular design, the framework introduces over 30 new modules.
Temporal modules' names starting with t./t.rast/t.vect/t.rast3d names
comply with the GRASS modules' naming conventions.

\subsection{Implementation}
Temporal framework uses a snapshot approach (described above).
This approach can be easily understood and is simple enough to be integrated into the layer-based GIS.
The integration consists of two levels.
In the first level, timestamps are assigned to the existing spatial datatypes -- raster, vector, 3D raster maps.
The second level introduces new datatypes -- space time raster,
vector and 3D raster datasets (refered as stdrs, stvds and str3ds).

The implementation of the temporal framework is based on a temporal library and a SQL database scheme.
The library is written in Python programming language and provides an application programming interface (API) which is used by the temporal modules,
however the library is meant to be used not only by other modules but also within user scripts or the GRASS GUI if needed.
Beside valid time the temporal database stores also transaction time%
\footnote{Valid time is the time period during which a database fact is valid in the modeled reality,
transaction time  is the time period during which a database fact is stored in the database \cite{temporalGlossary}}
and other map and dataset metadata.
The temporal framework supports two different database backends -- SQLite%
\footnote{\url{http://www.sqlite.org/}} (more lightweight) and
PostgreSQL\footnote{\url{http://www.postgresql.org/}}.
However, GRASS GIS users usually do not have to come in contact with the underlying database backend
as the default SQLite driver is often what they need.

\subsection{Basic concepts}
The temporal framework follows the concept of linear and discrete time.
One can decide which time model to use -- interval time or time instances (also called point time).
Point time is a single moment in the time dimension
while interval is a period of time consisting of two instances -- start time instance and end time instance.
The time interval contains the start time but not the end time:
$$[start, end)$$

Each type is suitable for different types of data.
Consider temperature and precipitation measuring.
While temperature is measured in a given time instance and the measured value describes the state,
precipitation is measured over a given time period. The decision which model to choose is not always straighforward, however it appears that for many applications, interval time is a better choice \cite{pointVsInterval}.








\newpage

\bibliographystyle{ieeetr}
\bibliography{/home/anna/Documents/DP/bibtex.bib}

\end{document}
