\documentclass[czech,11pt,a4paper]{article}
\usepackage[utf8]{inputenc}
\usepackage{a4wide}
\usepackage[pdftex,breaklinks=true,colorlinks=true,urlcolor=blue,
  pagecolor=black,linkcolor=black]{hyperref}
\usepackage[czech]{babel}

\pagestyle{empty}

\renewcommand{\arraystretch}{1.3}

\begin{document}

\begin{center}
  {\Large Posudek vedoucího diplomové práce}
\end{center}

\vspace{.5cm}

\noindent \begin{tabular}{lp{.75\textwidth}}
  {\bf Diplomová práce:} &
  Vizualizace časoprostorových dat v systému GRASS \\
  {\bf Student:} & Bc. Anna Kratochvílová \\
  {\bf Vedoucí:} & Ing. Martin Landa \\
  {\bf Oponent:} & Dipl. Ing. S\"oren Gebbert \\
\end{tabular}

\vspace{1cm}

Cílem diplomové práce Anny Kratochvílové bylo navrhnout a
implementovat nástroje pro vizualizaci časoprostorových dat
pro geografický informační systém GRASS. \\

Text diplomové práce je napsán v anglickém jazyce a je členěn do tří
kapitol. První kapitola se věnuje teoretické části práce zaměřené na
časoprostorová data obecně a na způsoby jejich vizualizace. Kromě toho
je uveden i stručných přehled softwarových produktů z této
oblasti. Druhá kapitola je již zaměřena na prostředí systému GRASS a
to především na projekt {\em Temporal GRASS GIS Framework}. Současně
je uveden přehled aktuálně dostupných nástrojů určených pro
vizualizaci časoprostorových dat a práci s nimi v systému
GRASS. Sofistikovanou podporu pro časoprostorová data přináší až ve
verzi GRASS 7 již zmiňovaný projekt Temporal Framework. V tomto ohledu
práce Anny Kratochvílové lze označit za vysoce aktuální, sledující
současný vývoj podpory časoprostorových dat v systému GRASS a jeho
posun ke 4D GISu.

Poslední kapitola práce je již zaměřena na konkrétní výstupy práce --
nástroje \uv{Animation Tool}, \uv{Timeline Tool} a \uv{Map
  Swipe}. Vzniklé nástroje velmi úzce sledují aktuální vývoj projektu
Temporal GRASS GIS Framework a v mnohém na něj navazují. Výstupy
předkládané diplomové práce tento projekt významně rozšiřující právě o
sofistikované vizualizační nástroje. \\

Anna Kratochvílová patří mezi nejtalentovanější a nejpilnější
studenty, se kterými jsem se při působení na Fakultě stavební ČVUT v
Praze setkal. Diplomová práce zcela naplňuje očekávání, která jsem
jako školitel do této práce vkládal. Diplomantka pracovala zcela
samostatně, v problematice časoprostorových dat v GIS se zorientovala
velmi rychle, při návrhu a~vývoji nástrojů velmi úzce spolupracovala
přímo s autorem projektu Temporal GRASS GIS Framework S\"orenem
Gebbertem. V diplomové práci se odráží dlouhodobý zájem a zkušenosti
autorky s vývojem nástrojů pro GIS. Po úspěšném obhájení své
bakalářské práce v roce 2011 na téma vývoje grafického nástroje pro
tvorbu mapových výstupů v~GRASS GIS spolupracovala s kolegou Václavem
Petrášem mimo jiné na vývoji nástroje pro řízenou klasifikaci v~GRASS
GIS či zásuvného modulu QGIS pro katastrální data poskytovaná ve
výměnného formátu katastru nemovitostí. Její soustavná práce plynule
vyústila v~předkládanou diplomovou práci.

S diplomantkou jsem spolupracoval soustavně po většinu jejího studia
na oboru Geoinformatika. Tato spolupráce vedla ke znatelnému rozšíření
funkcionality systému GRASS hned v~několika oblastech. Zpětně hodnotím
tuto spolupráci velmi kladně a jsem rád, že ji bylo možno realizovat v
takovém rozsahu. \newpage

Autorka diplomové práce nejen splnila svůj cíl -- "implementaci
nástrojů pro vizualizaci časoprostorových dat v systému GRASS" -- ale
dokázala ho znatelně překročit -- výsledek práce byl již začleněn do
projektu GRASS a je v současnosti testován společně s Temporal
Frameworkem. Práce plně splňuje požadavky kladené na diplomovou práci
na studijním programu Geodézie a kartografie včetně všech formálních
náležitostí a podle mého názoru je výrazně překračuje.

Vzhledem k úrovni zpracování práce hodnotím diplomovou práci Anny
Kratochvílové stupněm

\begin{center}
{\bf --- A (výborně)  ---}
\end{center}

a doporučuji státnicové komisi, aby zvážila možnost podání návrhu
děkance fakulty na udělení pochvaly za vynikající zpracování, obhajobu
a přínos diplomové práce.

\vspace{2cm}

\begin{tabular}{lp{.3\textwidth}c}
V~Praze dne \today & & \ldots\ldots\ldots\ldots\ldots\ldots\ldots \\
& & Ing. Martin Landa \\
& & Fakulta stavební \\
& & ČVUT v Praze \\
\end{tabular}

\end{document}
